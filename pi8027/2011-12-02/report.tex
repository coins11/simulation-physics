\documentclass{jsarticle}

\usepackage{color}
\usepackage[dvips]{graphicx}
\usepackage{listings}

\lstset{
  basicstyle=\small\tt,
  keywordstyle=\bf,
  identifierstyle=,
  commentstyle=,
  stringstyle=,
  emphstyle=,
  backgroundcolor=\color[gray]{0.8},
  language=Haskell,
  frame=trbl,
  rulecolor=\color{white},
  numbers=left,
  numberstyle=,
  basewidth={0.54em, 0.45em},
  lineskip=-0.2zw
}

\title{シミュレーション物理 レポート}
\author{情報科学類 201111365 坂口和彦}
\date{2011-12-09}

\begin{document}

\maketitle

\section{練習問題(2)}

この問題では、リスト\ref{list:prog2}のHaskellプログラムを用いた。

\begin{lstlisting}[label=list:prog2, caption=練習問題(2)のプログラム]
module Main where

import Control.Monad
import System.Random

points :: Int
points = 1000

rand :: IO Int
rand = getStdRandom (randomR (-1000000, 1000000))

main :: IO ()
main = do
    r <- replicateM points (liftM2 (,) rand rand)
    let len = length $ filter (\(x, y) -> (x*x + y*y) < 1000000000000) r
    let r = fromIntegral (4 * len) / fromIntegral points
    print $ r
    print $ (pi - r) / pi
\end{lstlisting}

結果は表\ref{table:prog2}の通りである。

\begin{table}[htb]

\begin{center}
\begin{tabular}{|l|l|l|l|}\hline
点の数                              & 100     & 1000      & 10000    \\\hline
面積の平均値(10回の平均)            & 3.052   & 3.1428    & 3.14736  \\\hline
$\frac{\pi - \textrm{平均値}}{\pi}$ & 0.02851 & 0.0003843 & 0.001835 \\\hline
\end{tabular}
\end{center}

\label{table:prog2}

\caption{練習問題(2)の結果}

\end{table}

\subsection{考察}

乱数が完全に一様に分布していると仮定すれば、点の数が1000の時よりも10000の方がより円周率に近い値が得られるはずだが、結果は異なったものであった。

このことから、乱数の精度の問題で10000回(10回の平均なので)以上値を取ったとしてもより良い結果が出るわけではないということが分かる。

より良い結果を得たいのであれば、完全に等間隔に点を並べて、かつその間隔を狭めていくのが適切な方法である。

\section{練習問題(3)}

この問題では、リスト\ref{list:prog3}のHaskellプログラムとgnuplotを用いた。

\begin{lstlisting}[label=list:prog3, caption=練習問題(3)のプログラム]
module Main where

import Control.Monad
import System.Random

points :: Int
points = 200

rand :: IO Int
rand = getStdRandom (randomR (-1000000, 1000000))

getPoint :: IO (Float, Float)
getPoint = do
    x <- rand
    y <- rand
    if x*x + y*y < 1000000000000
        then return (fromIntegral x / 1000000, fromIntegral y / 1000000)
        else getPoint

main :: IO ()
main = do
    r <- replicateM points getPoint
    mapM_ (\(x, y) -> putStrLn $ show x ++ " " ++ show y) r
\end{lstlisting}

結果は図\ref{fig:prog3}の通りである。

\begin{figure}[htbp]

\begin{center}
\includegraphics[width=100mm]{data.ps}
\end{center}

\caption{練習問題(3)の結果}
\label{fig:prog3}

\end{figure}

\end{document}
