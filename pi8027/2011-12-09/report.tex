\documentclass{jsarticle}

\usepackage{color}
\usepackage[dvips]{graphicx}
\usepackage{listings}
\usepackage{amsmath}
\usepackage{amssymb}

\lstdefinestyle{program}{
  basicstyle=\small\tt,
  keywordstyle=\bf,
  identifierstyle=,
  commentstyle=,
  stringstyle=,
  emphstyle=,
  backgroundcolor=\color[gray]{0.8},
  language=Haskell,
  frame=trbl,
  rulecolor=\color{white},
  numbers=left,
  numberstyle=,
  basewidth={0.54em, 0.45em},
  lineskip=-0.2zw
}

\title{シミュレーション物理 レポート}
\author{情報科学類 201111365 坂口和彦}
\date{2011-12-16}

\begin{document}

\maketitle

\section{練習問題(5)}

この問題では、リスト\ref{list:prog5}のHaskellプログラムを用いた。

\begin{lstlisting}[style=program, label=list:prog5, caption=練習問題(5)のプログラム]
import Data.Maybe
import Data.Bits
import qualified Data.Map as M

step rule l = maybe [] (\(h, t) -> step' t h l) (headLast l) where
    headLast [] = Nothing
    headLast (x : xs) = Just (x, foldl (const id) x xs)
    cell k = fromMaybe False $ M.lookup k rule
    step' h t [] = []
    step' h t [x] = [cell (h, x, t)]
    step' h t (x : xs@(x' : _)) = cell (h, x, x') : step' x t xs

makeRule n = M.fromList
    [((x, y, z), odd (shiftR n (4*xn + 2*yn + zn))) |
        (x, xn) <- pl, (y, yn) <- pl, (z, zn) <- pl]
    where pl = [(False, 0), (True, 1)]

runAuto n = iterate (step (makeRule n))
\end{lstlisting}

\subsection{結果}

図\ref{fig:output1}、図\ref{fig:output2}、図\ref{fig:output3}
に初期密度=0.3, 0.5, 0.7のときの空間時間図の例を示す。

\begin{figure}[htbp]
  \begin{align*}
    t = 0  & : 0 ~ 0 ~ 1 ~ 0 ~ 0 ~ 0 ~ 0 ~ 1 ~ 1 ~ 1 ~ 0 ~ 0 ~ 0 ~ 0 ~ 1 ~ 0 ~ 0 ~ 0 ~ 0 ~ 1 \\
    t = 1  & : 1 ~ 0 ~ 0 ~ 1 ~ 0 ~ 0 ~ 0 ~ 1 ~ 1 ~ 0 ~ 1 ~ 0 ~ 0 ~ 0 ~ 0 ~ 1 ~ 0 ~ 0 ~ 0 ~ 0 \\
    t = 2  & : 0 ~ 1 ~ 0 ~ 0 ~ 1 ~ 0 ~ 0 ~ 1 ~ 0 ~ 1 ~ 0 ~ 1 ~ 0 ~ 0 ~ 0 ~ 0 ~ 1 ~ 0 ~ 0 ~ 0 \\
    t = 3  & : 0 ~ 0 ~ 1 ~ 0 ~ 0 ~ 1 ~ 0 ~ 0 ~ 1 ~ 0 ~ 1 ~ 0 ~ 1 ~ 0 ~ 0 ~ 0 ~ 0 ~ 1 ~ 0 ~ 0 \\
    t = 4  & : 0 ~ 0 ~ 0 ~ 1 ~ 0 ~ 0 ~ 1 ~ 0 ~ 0 ~ 1 ~ 0 ~ 1 ~ 0 ~ 1 ~ 0 ~ 0 ~ 0 ~ 0 ~ 1 ~ 0 \\
    t = 5  & : 0 ~ 0 ~ 0 ~ 0 ~ 1 ~ 0 ~ 0 ~ 1 ~ 0 ~ 0 ~ 1 ~ 0 ~ 1 ~ 0 ~ 1 ~ 0 ~ 0 ~ 0 ~ 0 ~ 1 \\
    t = 6  & : 1 ~ 0 ~ 0 ~ 0 ~ 0 ~ 1 ~ 0 ~ 0 ~ 1 ~ 0 ~ 0 ~ 1 ~ 0 ~ 1 ~ 0 ~ 1 ~ 0 ~ 0 ~ 0 ~ 0 \\
    t = 7  & : 0 ~ 1 ~ 0 ~ 0 ~ 0 ~ 0 ~ 1 ~ 0 ~ 0 ~ 1 ~ 0 ~ 0 ~ 1 ~ 0 ~ 1 ~ 0 ~ 1 ~ 0 ~ 0 ~ 0 \\
    t = 8  & : 0 ~ 0 ~ 1 ~ 0 ~ 0 ~ 0 ~ 0 ~ 1 ~ 0 ~ 0 ~ 1 ~ 0 ~ 0 ~ 1 ~ 0 ~ 1 ~ 0 ~ 1 ~ 0 ~ 0 \\
    t = 9  & : 0 ~ 0 ~ 0 ~ 1 ~ 0 ~ 0 ~ 0 ~ 0 ~ 1 ~ 0 ~ 0 ~ 1 ~ 0 ~ 0 ~ 1 ~ 0 ~ 1 ~ 0 ~ 1 ~ 0 \\
    t = 10 & : 0 ~ 0 ~ 0 ~ 0 ~ 1 ~ 0 ~ 0 ~ 0 ~ 0 ~ 1 ~ 0 ~ 0 ~ 1 ~ 0 ~ 0 ~ 1 ~ 0 ~ 1 ~ 0 ~ 1 \\
    t = 11 & : 1 ~ 0 ~ 0 ~ 0 ~ 0 ~ 1 ~ 0 ~ 0 ~ 0 ~ 0 ~ 1 ~ 0 ~ 0 ~ 1 ~ 0 ~ 0 ~ 1 ~ 0 ~ 1 ~ 0 \\
    t = 12 & : 0 ~ 1 ~ 0 ~ 0 ~ 0 ~ 0 ~ 1 ~ 0 ~ 0 ~ 0 ~ 0 ~ 1 ~ 0 ~ 0 ~ 1 ~ 0 ~ 0 ~ 1 ~ 0 ~ 1 \\
    t = 13 & : 1 ~ 0 ~ 1 ~ 0 ~ 0 ~ 0 ~ 0 ~ 1 ~ 0 ~ 0 ~ 0 ~ 0 ~ 1 ~ 0 ~ 0 ~ 1 ~ 0 ~ 0 ~ 1 ~ 0 \\
    t = 14 & : 0 ~ 1 ~ 0 ~ 1 ~ 0 ~ 0 ~ 0 ~ 0 ~ 1 ~ 0 ~ 0 ~ 0 ~ 0 ~ 1 ~ 0 ~ 0 ~ 1 ~ 0 ~ 0 ~ 1 \\
    t = 15 & : 1 ~ 0 ~ 1 ~ 0 ~ 1 ~ 0 ~ 0 ~ 0 ~ 0 ~ 1 ~ 0 ~ 0 ~ 0 ~ 0 ~ 1 ~ 0 ~ 0 ~ 1 ~ 0 ~ 0 \\
    t = 16 & : 0 ~ 1 ~ 0 ~ 1 ~ 0 ~ 1 ~ 0 ~ 0 ~ 0 ~ 0 ~ 1 ~ 0 ~ 0 ~ 0 ~ 0 ~ 1 ~ 0 ~ 0 ~ 1 ~ 0 \\
    t = 17 & : 0 ~ 0 ~ 1 ~ 0 ~ 1 ~ 0 ~ 1 ~ 0 ~ 0 ~ 0 ~ 0 ~ 1 ~ 0 ~ 0 ~ 0 ~ 0 ~ 1 ~ 0 ~ 0 ~ 1 \\
    t = 18 & : 1 ~ 0 ~ 0 ~ 1 ~ 0 ~ 1 ~ 0 ~ 1 ~ 0 ~ 0 ~ 0 ~ 0 ~ 1 ~ 0 ~ 0 ~ 0 ~ 0 ~ 1 ~ 0 ~ 0 \\
    t = 19 & : 0 ~ 1 ~ 0 ~ 0 ~ 1 ~ 0 ~ 1 ~ 0 ~ 1 ~ 0 ~ 0 ~ 0 ~ 0 ~ 1 ~ 0 ~ 0 ~ 0 ~ 0 ~ 1 ~ 0 \\
  \end{align*}
  \caption{空間時間図(初期密度=0.3)}
  \label{fig:output1}
\end{figure}

\begin{figure}[htbp]
  \begin{align*}
    t = 0  & : 1 ~ 0 ~ 1 ~ 0 ~ 1 ~ 1 ~ 1 ~ 0 ~ 0 ~ 1 ~ 1 ~ 0 ~ 0 ~ 0 ~ 1 ~ 1 ~ 0 ~ 0 ~ 0 ~ 1 \\
    t = 1  & : 0 ~ 1 ~ 0 ~ 1 ~ 1 ~ 1 ~ 0 ~ 1 ~ 0 ~ 1 ~ 0 ~ 1 ~ 0 ~ 0 ~ 1 ~ 0 ~ 1 ~ 0 ~ 0 ~ 1 \\
    t = 2  & : 1 ~ 0 ~ 1 ~ 1 ~ 1 ~ 0 ~ 1 ~ 0 ~ 1 ~ 0 ~ 1 ~ 0 ~ 1 ~ 0 ~ 0 ~ 1 ~ 0 ~ 1 ~ 0 ~ 0 \\
    t = 3  & : 0 ~ 1 ~ 1 ~ 1 ~ 0 ~ 1 ~ 0 ~ 1 ~ 0 ~ 1 ~ 0 ~ 1 ~ 0 ~ 1 ~ 0 ~ 0 ~ 1 ~ 0 ~ 1 ~ 0 \\
    t = 4  & : 0 ~ 1 ~ 1 ~ 0 ~ 1 ~ 0 ~ 1 ~ 0 ~ 1 ~ 0 ~ 1 ~ 0 ~ 1 ~ 0 ~ 1 ~ 0 ~ 0 ~ 1 ~ 0 ~ 1 \\
    t = 5  & : 1 ~ 1 ~ 0 ~ 1 ~ 0 ~ 1 ~ 0 ~ 1 ~ 0 ~ 1 ~ 0 ~ 1 ~ 0 ~ 1 ~ 0 ~ 1 ~ 0 ~ 0 ~ 1 ~ 0 \\
    t = 6  & : 1 ~ 0 ~ 1 ~ 0 ~ 1 ~ 0 ~ 1 ~ 0 ~ 1 ~ 0 ~ 1 ~ 0 ~ 1 ~ 0 ~ 1 ~ 0 ~ 1 ~ 0 ~ 0 ~ 1 \\
    t = 7  & : 0 ~ 1 ~ 0 ~ 1 ~ 0 ~ 1 ~ 0 ~ 1 ~ 0 ~ 1 ~ 0 ~ 1 ~ 0 ~ 1 ~ 0 ~ 1 ~ 0 ~ 1 ~ 0 ~ 1 \\
    t = 8  & : 1 ~ 0 ~ 1 ~ 0 ~ 1 ~ 0 ~ 1 ~ 0 ~ 1 ~ 0 ~ 1 ~ 0 ~ 1 ~ 0 ~ 1 ~ 0 ~ 1 ~ 0 ~ 1 ~ 0 \\
    t = 9  & : 0 ~ 1 ~ 0 ~ 1 ~ 0 ~ 1 ~ 0 ~ 1 ~ 0 ~ 1 ~ 0 ~ 1 ~ 0 ~ 1 ~ 0 ~ 1 ~ 0 ~ 1 ~ 0 ~ 1 \\
    t = 10 & : 1 ~ 0 ~ 1 ~ 0 ~ 1 ~ 0 ~ 1 ~ 0 ~ 1 ~ 0 ~ 1 ~ 0 ~ 1 ~ 0 ~ 1 ~ 0 ~ 1 ~ 0 ~ 1 ~ 0 \\
    t = 11 & : 0 ~ 1 ~ 0 ~ 1 ~ 0 ~ 1 ~ 0 ~ 1 ~ 0 ~ 1 ~ 0 ~ 1 ~ 0 ~ 1 ~ 0 ~ 1 ~ 0 ~ 1 ~ 0 ~ 1 \\
    t = 12 & : 1 ~ 0 ~ 1 ~ 0 ~ 1 ~ 0 ~ 1 ~ 0 ~ 1 ~ 0 ~ 1 ~ 0 ~ 1 ~ 0 ~ 1 ~ 0 ~ 1 ~ 0 ~ 1 ~ 0 \\
    t = 13 & : 0 ~ 1 ~ 0 ~ 1 ~ 0 ~ 1 ~ 0 ~ 1 ~ 0 ~ 1 ~ 0 ~ 1 ~ 0 ~ 1 ~ 0 ~ 1 ~ 0 ~ 1 ~ 0 ~ 1 \\
    t = 14 & : 1 ~ 0 ~ 1 ~ 0 ~ 1 ~ 0 ~ 1 ~ 0 ~ 1 ~ 0 ~ 1 ~ 0 ~ 1 ~ 0 ~ 1 ~ 0 ~ 1 ~ 0 ~ 1 ~ 0 \\
    t = 15 & : 0 ~ 1 ~ 0 ~ 1 ~ 0 ~ 1 ~ 0 ~ 1 ~ 0 ~ 1 ~ 0 ~ 1 ~ 0 ~ 1 ~ 0 ~ 1 ~ 0 ~ 1 ~ 0 ~ 1 \\
    t = 16 & : 1 ~ 0 ~ 1 ~ 0 ~ 1 ~ 0 ~ 1 ~ 0 ~ 1 ~ 0 ~ 1 ~ 0 ~ 1 ~ 0 ~ 1 ~ 0 ~ 1 ~ 0 ~ 1 ~ 0 \\
    t = 17 & : 0 ~ 1 ~ 0 ~ 1 ~ 0 ~ 1 ~ 0 ~ 1 ~ 0 ~ 1 ~ 0 ~ 1 ~ 0 ~ 1 ~ 0 ~ 1 ~ 0 ~ 1 ~ 0 ~ 1 \\
    t = 18 & : 1 ~ 0 ~ 1 ~ 0 ~ 1 ~ 0 ~ 1 ~ 0 ~ 1 ~ 0 ~ 1 ~ 0 ~ 1 ~ 0 ~ 1 ~ 0 ~ 1 ~ 0 ~ 1 ~ 0 \\
    t = 19 & : 0 ~ 1 ~ 0 ~ 1 ~ 0 ~ 1 ~ 0 ~ 1 ~ 0 ~ 1 ~ 0 ~ 1 ~ 0 ~ 1 ~ 0 ~ 1 ~ 0 ~ 1 ~ 0 ~ 1 \\
  \end{align*}
  \caption{空間時間図(初期密度=0.5)}
  \label{fig:output2}
\end{figure}

\begin{figure}[htbp]
  \begin{align*}
    t = 0  & : 1 ~ 1 ~ 0 ~ 1 ~ 1 ~ 1 ~ 0 ~ 0 ~ 0 ~ 1 ~ 1 ~ 1 ~ 1 ~ 1 ~ 1 ~ 1 ~ 0 ~ 0 ~ 1 ~ 1 \\
    t = 1  & : 1 ~ 0 ~ 1 ~ 1 ~ 1 ~ 0 ~ 1 ~ 0 ~ 0 ~ 1 ~ 1 ~ 1 ~ 1 ~ 1 ~ 1 ~ 0 ~ 1 ~ 0 ~ 1 ~ 1 \\
    t = 2  & : 0 ~ 1 ~ 1 ~ 1 ~ 0 ~ 1 ~ 0 ~ 1 ~ 0 ~ 1 ~ 1 ~ 1 ~ 1 ~ 1 ~ 0 ~ 1 ~ 0 ~ 1 ~ 1 ~ 1 \\
    t = 3  & : 1 ~ 1 ~ 1 ~ 0 ~ 1 ~ 0 ~ 1 ~ 0 ~ 1 ~ 1 ~ 1 ~ 1 ~ 1 ~ 0 ~ 1 ~ 0 ~ 1 ~ 1 ~ 1 ~ 0 \\
    t = 4  & : 1 ~ 1 ~ 0 ~ 1 ~ 0 ~ 1 ~ 0 ~ 1 ~ 1 ~ 1 ~ 1 ~ 1 ~ 0 ~ 1 ~ 0 ~ 1 ~ 1 ~ 1 ~ 0 ~ 1 \\
    t = 5  & : 1 ~ 0 ~ 1 ~ 0 ~ 1 ~ 0 ~ 1 ~ 1 ~ 1 ~ 1 ~ 1 ~ 0 ~ 1 ~ 0 ~ 1 ~ 1 ~ 1 ~ 0 ~ 1 ~ 1 \\
    t = 6  & : 0 ~ 1 ~ 0 ~ 1 ~ 0 ~ 1 ~ 1 ~ 1 ~ 1 ~ 1 ~ 0 ~ 1 ~ 0 ~ 1 ~ 1 ~ 1 ~ 0 ~ 1 ~ 1 ~ 1 \\
    t = 7  & : 1 ~ 0 ~ 1 ~ 0 ~ 1 ~ 1 ~ 1 ~ 1 ~ 1 ~ 0 ~ 1 ~ 0 ~ 1 ~ 1 ~ 1 ~ 0 ~ 1 ~ 1 ~ 1 ~ 0 \\
    t = 8  & : 0 ~ 1 ~ 0 ~ 1 ~ 1 ~ 1 ~ 1 ~ 1 ~ 0 ~ 1 ~ 0 ~ 1 ~ 1 ~ 1 ~ 0 ~ 1 ~ 1 ~ 1 ~ 0 ~ 1 \\
    t = 9  & : 1 ~ 0 ~ 1 ~ 1 ~ 1 ~ 1 ~ 1 ~ 0 ~ 1 ~ 0 ~ 1 ~ 1 ~ 1 ~ 0 ~ 1 ~ 1 ~ 1 ~ 0 ~ 1 ~ 0 \\
    t = 10 & : 0 ~ 1 ~ 1 ~ 1 ~ 1 ~ 1 ~ 0 ~ 1 ~ 0 ~ 1 ~ 1 ~ 1 ~ 0 ~ 1 ~ 1 ~ 1 ~ 0 ~ 1 ~ 0 ~ 1 \\
    t = 11 & : 1 ~ 1 ~ 1 ~ 1 ~ 1 ~ 0 ~ 1 ~ 0 ~ 1 ~ 1 ~ 1 ~ 0 ~ 1 ~ 1 ~ 1 ~ 0 ~ 1 ~ 0 ~ 1 ~ 0 \\
    t = 12 & : 1 ~ 1 ~ 1 ~ 1 ~ 0 ~ 1 ~ 0 ~ 1 ~ 1 ~ 1 ~ 0 ~ 1 ~ 1 ~ 1 ~ 0 ~ 1 ~ 0 ~ 1 ~ 0 ~ 1 \\
    t = 13 & : 1 ~ 1 ~ 1 ~ 0 ~ 1 ~ 0 ~ 1 ~ 1 ~ 1 ~ 0 ~ 1 ~ 1 ~ 1 ~ 0 ~ 1 ~ 0 ~ 1 ~ 0 ~ 1 ~ 1 \\
    t = 14 & : 1 ~ 1 ~ 0 ~ 1 ~ 0 ~ 1 ~ 1 ~ 1 ~ 0 ~ 1 ~ 1 ~ 1 ~ 0 ~ 1 ~ 0 ~ 1 ~ 0 ~ 1 ~ 1 ~ 1 \\
    t = 15 & : 1 ~ 0 ~ 1 ~ 0 ~ 1 ~ 1 ~ 1 ~ 0 ~ 1 ~ 1 ~ 1 ~ 0 ~ 1 ~ 0 ~ 1 ~ 0 ~ 1 ~ 1 ~ 1 ~ 1 \\
    t = 16 & : 0 ~ 1 ~ 0 ~ 1 ~ 1 ~ 1 ~ 0 ~ 1 ~ 1 ~ 1 ~ 0 ~ 1 ~ 0 ~ 1 ~ 0 ~ 1 ~ 1 ~ 1 ~ 1 ~ 1 \\
    t = 17 & : 1 ~ 0 ~ 1 ~ 1 ~ 1 ~ 0 ~ 1 ~ 1 ~ 1 ~ 0 ~ 1 ~ 0 ~ 1 ~ 0 ~ 1 ~ 1 ~ 1 ~ 1 ~ 1 ~ 0 \\
    t = 18 & : 0 ~ 1 ~ 1 ~ 1 ~ 0 ~ 1 ~ 1 ~ 1 ~ 0 ~ 1 ~ 0 ~ 1 ~ 0 ~ 1 ~ 1 ~ 1 ~ 1 ~ 1 ~ 0 ~ 1 \\
    t = 19 & : 1 ~ 1 ~ 1 ~ 0 ~ 1 ~ 1 ~ 1 ~ 0 ~ 1 ~ 0 ~ 1 ~ 0 ~ 1 ~ 1 ~ 1 ~ 1 ~ 1 ~ 0 ~ 1 ~ 0 \\
  \end{align*}
  \caption{空間時間図(初期密度=0.7)}
  \label{fig:output3}
\end{figure}

\subsection{考察}

実際の自動車道の振舞いのように、車間距離があいていなければ前に進めないルールとなっているため、
交通流シミュレーションをするのに相応しい振舞いをしている。実際に、以下のような事柄が観察できる。

\begin{itemize}
\item 初期密度が低ければ最初の何ステップかで車間距離を取り、一定の速度で進む。
\item 初期密度が0.5であれば、途中から0と1が交互に出現するような並びになる。
\item 初期密度が高ければ空白となっている箇所が前に来るまで進めず、まさに渋滞のような状況となっている。
\end{itemize}

\end{document}
